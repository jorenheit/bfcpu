\section{Introduction}
The Brainf*ck\footnote{The asterisk was inserted by the authors and is not part of the official name.} (BF) programming language is an esoteric programming language that is basically impossible, or at least very unpractical, to actually write useful programs in. Even if you would become a very skilled programmer in this language, the resulting programs would be incredibly slow to execute. Despite this, many programmers have challenged themselves to write stunning pieces of code just for fun, or for the learning experience it offers. In doing so, it teaches us about computer architecture, compilers/interpreters, memory, pointers and much more. For more information on the language itself, see chapter \ref{section:brainfck}.

The goal of this project is to build a computer that can actually run BF code natively. Normally, after having written some new piece of BF, the programmer must run this code in another program to either compile or interpret it to see whether or not it works. However, when looking more closely at the language, it seems a lot like an instruction set for a (not yet) existing processor. This is what we aim to do in this project: build that thing.

The aim is to build a Brainf*ck CPU without making use of any programmable chips. We will only use discrete logic and relatively simple integrated circuits like registers, buffers and (de)multiplexers. The computer will be built on breadboards, as it was inspired by Ben Eater's 8-bit breadboard computer \cite{beneater}. This report will document our building proces.
