\section{Preface} \label{sec:preface}
This documentation will encapsulate underlying principles of the processor based on brainf*ck programming language. Even though the documentation does not dive deeply in electronics or underlying science of computers, there are some recommendations for readers to provide the most satisfying reading process. 

A reader should be familiar with:
\begin{itemize}
	\item some elements of fundamental electronics such as capacitors, resistors, filters, voltage biasing, and basic electromagnetism;
	\item intermediate understanding of digital electronics and boolean algebra;
	\item comprehension of programming language(s), such as C/C++ or brainf*ck.
\end{itemize}

\subsection{Useful resources}
If the reader does not meet certain prerequisites, the following references may be helpful for acquaintance:
\begin{itemize}
	\item "Digital Electronics by Morris Mano" for basic digital electronics and boolean algebra;
	\item "The Feynman Lectures on Physics Volume 2" for introductory electromagnetism;
	\item "Introduction to Electrodynamics by David J. Griffiths" for intermediate electrodynamics and applications;
	\item "learncpp.com" is a great learning resource for C++;
	\item for further resources see \ref{resources}
\end{itemize}
