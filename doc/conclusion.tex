\section{Conclusion} \label{sec:conclusion}
This project set out as a ``What if?'' for which expectations were deliberately kept low. A functional system was the goal; not a fast one. We had no idea what kinds of speeds we could expect, how stable the system would be and frankly whether we would even end up with anything functional. The only goal was simply to get something going. Given those goals, this project has certainly delivered to an unexpected extent.

Building Synapse-191 has been a tremendously powerful learning experience. Above all, the joy of seeing months of tedious work, planning and debugging culminate into a working system is unparalleled. The system's specifications itself never prevented any of the tested BF programs to run; it was never low on memory, has a large enough program-ROM to store BF code and has plenty of stack-space to run even computationally demanding programs. Features like the IO system, being able to store multiple programs to select from, single-stepping and easy control of the clock make it an enjoyable process to operate the computer. Moreover, with the toolchain developed to a mature state, adding features through modification of the firmware and microcode was relatively straightforward and thoroughly rewarding to do.

The system has at the time of writing been framed and hung from a wall. External ports have been added to the frame such that even when hanging, it can still easily be plugged in and controled using the keyboard. Some 3D printed parts and have been added for visual effects, but no further extensions will be added to the system as-is. There are currently no plans for future development of a next generation (e.g.~on a PCB), because \emph{what's the point}? Then again, what \emph{was} the point, ever?

\vfill
\begin{center}
  \includegraphics[width=.9\textwidth]{img/framed}
\end{center}


